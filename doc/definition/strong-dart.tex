\documentclass[fleqn, draft]{article}
\usepackage{proof, amsmath, amssymb, ifthen}
%%% Cascaded items for math mode
%% start with \begin{cascade}
%% new line at previous indentation with \cascline
%% new line with greater indentation with \cascitem
%% end with \end{cascade}
%% default indentation is 2em, adjust with \cascadeindent
\newdimen\cascadeindent
\cascadeindent=1em\newdimen\cascdimen
\newcommand{\cascindent}{\global\advance\cascdimen by\cascadeindent \hspace{\cascdimen}}
\newcommand{\cascitem}{\\ \global\advance\cascdimen by\cascadeindent \hspace{\cascdimen}}
\newcommand{\cascback}[1]{\\ \global\advance\cascdimen by-#1.0\cascadeindent \hspace{\cascdimen}}
\newcommand{\cascline}{\\ \hspace{\cascdimen}}
\newenvironment{cascade}{\begin{array}[t]{@{}l@{}} \global\cascdimen=0em}{\end{array}}


%%% Binding colon stuff
\mathchardef\col="003A  % \col for binding colon (mathcode ordinary: less space)
\mathchardef\semi="603B % \semi for (regular) semicolon
%% use \semicolonforbindingcolon to redefine ; to stand for binding colon
\newcommand{\semicolonforbindingcolon}{\mathcode`;="003A}

%%% Angle bracket stuff
\mathchardef\lt="313C  % \lt for <
\mathchardef\gt="313E  % \gt for >
%% use \ltgtforanglebrackets to redefine <,> to stand for \langle, \rangle
\newcommand{\ltgtforanglebrackets}{\mathcode`<="4268 \mathcode`>="5269}

\newcommand{\kwop}[1]{\ensuremath{\mathop{\mathbf{#1}}}}
\newcommand{\kwbin}[1]{\ensuremath{\mathbin{\mathbf{#1}}}}
\newcommand{\kw}[1]{\ensuremath{\mathord{\mathbf{#1}}}}

\newcommand{\comment}[1]{\hfill \fbox{\Large{#1}}}

%\newcommand{\qed}{\rule{5pt}{8pt}} 
\newcommand{\thmbox}
   {{\ \hfill\hbox{%
      \vrule width1.0ex height1.0ex
   }\parfillskip 0pt}}

\newenvironment{proof}{{\textbf{Proof:} }}{\thmbox}
\newenvironment{proofsketch}{{\textbf{Proof (Sketch):} }}{\thmbox}

\newcommand{\thmstep}[2]{
  \noindent\begin{tabular}{@{}l@{}l}
    \lefteqn{\mbox{#1}} &\\
    \mbox{  } & $\begin{array}{l}#2\end{array}$
  \end{tabular}
  }

\newcommand{\thmstepp}[2]{
  \noindent\begin{tabular}{lll}
    \lefteqn{\mbox{#1}} &\\
    \mbox{  } & #2
  \end{tabular}
  }

\newcommand{\ifthenthm}[2]{
  \noindent\begin{tabular}[t]{@{}l@{}l}
    If & \\
    & $\begin{array}[t]{l}#1\end{array}$ \\
    then & \\
    & $\begin{array}[t]{l}#2\end{array}$
  \end{tabular}
  }

% symbol abbreviations
\newcommand{\stepsto}{\longmapsto}


% types
\newcommand{\Arrow}[3][-]{#2 \overset{#1}{\rightarrow} #3}
\newcommand{\Bool}{\mathbf{bool}}
\newcommand{\Bottom}{\mathbf{bottom}}
\newcommand{\Dynamic}{\mathbf{dynamic}}
\newcommand{\Null}{\mathbf{Null}}
\newcommand{\Num}{\mathbf{num}}
\newcommand{\Object}{\mathbf{Object}}
\newcommand{\TApp}[2]{#1\mathrm{<}#2\mathrm{>}}
\newcommand{\Type}{\mathbf{Type}}
\newcommand{\Weak}[1]{\mathbf{\{#1\}}}
\newcommand{\Sig}{\mathit{Sig}}
\newcommand{\Boxed}[1]{\langle #1 \rangle}

% expressions
\newcommand{\eassign}[2]{#1 = #2}
\newcommand{\eas}[2]{#1\ \mathbf{as}\ #2}
\newcommand{\ebox}[2]{\langle#1\rangle_{#2}}
\newcommand{\ecall}[2]{#1(#2)}
\newcommand{\echeck}[2]{\kwop{check}(#1, #2)}
\newcommand{\edcall}[2]{\kwop{dcall}(#1, #2)}
\newcommand{\edload}[2]{\kwop{dload}(#1, #2)}
\newcommand{\edo}[1]{\kwdo\{\,#1\,\}}
\newcommand{\eff}{\mathrm{ff}}
\newcommand{\eis}[2]{#1\ \mathbf{is}\ #2}
\newcommand{\elabel}[1][l]{\mathit{l}}
\newcommand{\elambda}[3]{(#1):#2 \Rightarrow #3}
\newcommand{\eload}[2]{#1.#2}
\newcommand{\enew}[3]{\mathbf{new}\,\TApp{#1}{#2}(#3)}
\newcommand{\enull}{\mathbf{null}}
\newcommand{\eobject}[2]{\kwobject_{#1} \{#2\}}
\newcommand{\eprimapp}[2]{\ecall{#1}{#2}}
\newcommand{\eprim}{\kwop{op}}
\newcommand{\esend}[3]{\ecall{\eload{#1}{#2}}{#3}}
\newcommand{\eset}[3]{\eassign{#1.#2}{#3}}
\newcommand{\esuper}{\mathbf{super}}
\newcommand{\ethis}{\mathbf{this}}
\newcommand{\ethrow}{\mathbf{throw}}
\newcommand{\ett}{\mathrm{tt}}
\newcommand{\eunbox}[1]{{*#1}}

% keywords
\newcommand{\kwclass}{\kw{class}}
\newcommand{\kwdo}{\kw{do}}
\newcommand{\kwelse}{\kw{else}}
\newcommand{\kwextends}{\kw{extends}}
\newcommand{\kwfun}{\kw{fun}}
\newcommand{\kwif}{\kw{if}}
\newcommand{\kwin}{\kw{in}}
\newcommand{\kwlet}{\kw{let}}
\newcommand{\kwobject}{\kw{object}}
\newcommand{\kwreturn}{\kw{return}}
\newcommand{\kwthen}{\kw{then}}
\newcommand{\kwvar}{\kw{var}}

% declarations
\newcommand{\dclass}[3]{\kwclass\ #1\ \kwextends\ #2\ \{#3\}}
\newcommand{\dfun}[4]{#2(#3):#1 = #4}
\newcommand{\dvar}[2]{\kwvar\ #1\ =\ #2}


\newcommand{\fieldDecl}[2]{\kwvar\ #1 : #2}
\newcommand{\methodDecl}[3]{\kwfun\ #1 : \iftrans{#2 \triangleleft} #3}

% statements
\newcommand{\sifthenelse}[3]{\kwif\ (#1)\ \kwthen\ #2\ \kwelse\ #3}
\newcommand{\sreturn}[1]{\kwreturn\ #1}

% programs
\newcommand{\program}[2]{\kwlet\ #1\ \kwin\ #2}

% relational operators
\newcommand{\sub}{\mathbin{<:}}

% utilities
\newcommand{\many}[1]{\overrightarrow{#1}}
\newcommand{\alt}{\ \mathop{|}\ }
\newcommand{\opt}[1]{[#1]}
\newcommand{\bind}[3]{#1 \Leftarrow\, #2\ \kw{in}\ #3}

\newcommand{\note}[1]{\textbf{NOTE:} \textit{#1}}

%dynamic semantics
\newcommand{\TypeError}{\mathbf{Error}}

% inference rules
\newcommand{\infrulem}[3][]{
  \begin{array}{c@{\ }c}
    \begin{array}{cccc}
      #2 \vspace{-2mm} 
    \end{array} \\
    \hrulefill & #1 \\
    \begin{array}{l}
      #3
    \end{array}
  \end{array}
  }

\newcommand{\axiomm}[2][]{
  \begin{array}{cc}
    \hrulefill & #1 \\
    \begin{array}{c}
      #2
    \end{array}
  \end{array}
  }

\newcommand{\infrule}[3][]{
  \[ 
  \infrulem[#1]{#2}{#3}
  \]
  }

\newcommand{\axiom}[2][]{
  \[ 
  \axiomm[#1]{#2}
  \]
  }

% judgements and relations
\newboolean{show_translation}
\setboolean{show_translation}{false}
\newcommand{\iftrans}[1]{\ifthenelse{\boolean{show_translation}}{#1}{}}
\newcommand{\ifnottrans}[1]{\ifthenelse{\boolean{show_translation}}{#1}}

\newcommand{\blockOk}[4]{#1 \vdash #2 \col #3\iftrans{\, \Uparrow\, #4}}
\newcommand{\declOk}[5][]{#2 \vdash_{#1} #3 \, \Uparrow\, \iftrans{#4\, :\,} #5}
\newcommand{\extends}[4][:]{#2[#3\ #1\ #4]}
\newcommand{\fieldLookup}[4]{#1 \vdash #2.#3\, \leadsto_f\, #4}
\newcommand{\methodLookup}[5]{#1 \vdash #2.#3\, \leadsto_m\, \iftrans{#4 \triangleleft} #5}
\newcommand{\fieldAbsent}[3]{#1 \vdash #3 \notin #2}
\newcommand{\methodAbsent}[3]{#1 \vdash #3 \notin #2}
\newcommand{\hastype}[3]{#1 \vdash #2 \, : \, #3}
\newcommand{\stmtOk}[5]{#1 \vdash #2 \, : \, #3\, \Uparrow \iftrans{#4\, :\,} #5}
\newcommand{\subst}[2]{[#1/#2]}
\newcommand{\subtypeOfOpt}[5][?]{#2 \vdash\ #3 \sub^{#1} #4\, \Uparrow\, #5}
\newcommand{\subtypeOf}[4][]{#2 \vdash\ #3 \sub^{#1} #4}
\newcommand{\yieldsOk}[5]{#1 \vdash #2 \, : \, #3\, \Uparrow\, \iftrans{#4\, :\,} #5}
\newcommand{\programOk}[3]{#1 \vdash #2\iftrans{\, \Uparrow\, #3}}
\newcommand{\ok}[2]{#1 \vdash #2\, \mbox{\textbf{ok}}}
\newcommand{\overrideOk}[4]{#1 \vdash #2\,\kwextends\, #3 \Leftarrow\, #4}

\newcommand{\down}[1]{\ensuremath{\downharpoonleft\!\!#1\!\!\downharpoonright}}
\newcommand{\up}[1]{\ensuremath{\upharpoonleft\!\!#1\!\!\upharpoonright}}
\newcommand{\sigof}[1]{\mathit{sigof}(#1)}
\newcommand{\typeof}[1]{\mathit{typeof}(#1)}
\newcommand{\sstext}[2]{\ifthenelse{\boolean{show_translation}}{#2}{#1}}

\newcommand{\evaluatesTo}[5][]{\{#2\alt #3\}  \stepsto_{#1} \{#4 \alt #5\}}


\title{Dart strong mode definition}

\begin{document}

\textbf{\large PRELIMINARY DRAFT}

\section*{Syntax}


Terms and types.  Note that we allow types to be optional in certain positions
(currently function arguments and return types, and on variable declarations).
Implicitly these are either inferred or filled in with dynamic.

There are explicit terms for dynamic calls and loads, and for dynamic type
checks.

Fields can only be read or set within a method via a reference to this, so no
dynamic set operation is required (essentially dynamic set becomes a dynamic
call to a setter).  This just simplifies the presentation a bit.  Methods may be
externally loaded from the object (either to call them, or to pass them as
closurized functions).

\[
\begin{array}{lcl}
\text{Type identifiers} & ::= &  C, G, T, S, \ldots \\
%
\text{Arrow kind ($k$)} & ::= &  +, -\\
%
\text{Types $\tau, \sigma$} & ::= &
 T \alt \Dynamic \alt \Object \alt \Null \alt \Type \alt \Num \\ &&
   \alt \Bool
   \alt \Arrow[k]{\many{\tau}}{\sigma} \alt \TApp{C}{\many{\tau}} \\
%
\text{Ground types $\tau, \sigma$} & ::= &
 \Dynamic \alt \Object \alt \Null \alt \Type \alt \Num \\ &&
   \alt \Bool
   \alt \Arrow[+]{\many{\Dynamic}}{\Dynamic} \alt \TApp{C}{\many{\Dynamic}} \\
%
\text{Optional type ($[\tau]$)} & ::= &  \_ \alt \tau \\
%
\text{Term identifiers} & ::= & a, b, x, y, m, n, \ldots \\
%
\text{Primops ($\phi$)} & ::= & \mathrm{+}, \mathrm{-} \ldots \mathrm{||} \ldots \\
%
\text{Expressions $e$} & ::= & 
 x \alt i \alt \ett \alt \eff \alt \enull \alt \ethis \\&&
   \alt \elambda{\many{x:\opt{\tau}}}{\opt{\sigma}}{s} 
   \alt \enew{C}{\many{\tau}}{} \\&&
   \alt \eprimapp{\eprim}{\many{e}} \alt \ecall{e}{\many{e}} \\&&
   \alt \eload{e}{m} \alt \eload{\ethis}{x} \\&&
   \alt \eassign{x}{e} \alt \eset{\ethis}{x}{e} \\&&
   \alt \ethrow \alt \eas{e}{\tau} \alt \eis{e}{\tau} \\
%
\text{Declaration ($\mathit{vd}$)} & ::= &
   \dvar{x:\opt{\tau}}{e} \alt \dfun{\tau}{f}{\many{x:\tau}}{s} \\
%
\text{Statements ($s$)} & ::= & \mathit{vd} \alt e \alt \sifthenelse{e}{s_1}{s_2} 
   \alt \sreturn{e} \alt s;s \\
%
\text{Class decl ($\mathit{cd}$)} & ::= & \dclass{\TApp{C}{\many{T}}}{\TApp{G}{\many{\tau}}}{\many{\mathit{vd}}} \\
%
\text{Toplevel decl ($\mathit{td}$)} & ::= & \mathit{vd} \alt \mathit{cd}\\
%
\text{Program ($P$)} & ::= & \program{\many{\mathit{td}}}{s}
\end{array}
\]


Type contexts map type variables to their bounds.

Class signatures describe the methods and fields in an object, along with the
super class of the class.  There are no static methods or fields.

The class hierararchy records the classes with their signatures.

The term context maps term variables to their types.  I also abuse notation and
allow for the attachment of an optional type to term contexts as follows:
$\Gamma_\sigma$ refers to a term context within the body of a method whose class
type is $\sigma$.

\[
\begin{array}{lcl}
\text{Type context ($\Delta$)} & ::= &  \epsilon \alt \Delta, T \sub \tau \\
\text{Class element ($\mathit{ce}$)} & ::= & 
  \fieldDecl{x}{\tau} \alt \methodDecl{f}{\tau}{\sigma} \\
\text{Class signature ($\Sig$)} & ::= &
  \dclass{\TApp{C}{\many{T}}}{\TApp{G}{\many{\tau}}}{\many{\mathit{ce}}} \\
\text{Class hierarchy ($\Phi$)} & ::= &  \epsilon \alt \Phi, C\ :\ \Sig \\
\text{Term context ($\Gamma$)} & ::= &  \epsilon \alt \Gamma, x\ :\ \tau
\end{array}
\]


\section*{Subtyping}

\subsection*{Variant Subtyping}

We include a special kind of covariant function space to model certain dart
idioms.  An arrow type decorated with a positive variance annotation ($+$)
treats $\Dynamic$ in its argument list covariantly: or equivalently, it treats
$\Dynamic$ as bottom.  This variant subtyping relation captures this special
treatment of dynamic.

\axiom{\subtypeOf[+]{\Phi, \Delta}{\Dynamic}{\tau}}

\infrule{\subtypeOf{\Phi, \Delta}{\sigma}{\tau} \quad \sigma \neq \Dynamic}
        {\subtypeOf[+]{\Phi, \Delta}{\sigma}{\tau}}

\infrule{\subtypeOf{\Phi, \Delta}{\sigma}{\tau}}
        {\subtypeOf[-]{\Phi, \Delta}{\sigma}{\tau}}

\subsection*{Invariant Subtyping}

Regular subtyping is defined in a fairly standard way, except that generics are
uniformly covariant, and that function argument types fall into the variant
subtyping relation defined above.

\axiom{\subtypeOf{\Phi, \Delta}{\tau}{\Dynamic}}

\axiom{\subtypeOf{\Phi, \Delta}{\tau}{\Object}} 

\axiom{\subtypeOf{\Phi, \Delta}{\Bottom}{\tau}}

\axiom{\subtypeOf{\Phi, \Delta}{\tau}{\tau}} 

\infrule{(S\, :\, \sigma) \in \Delta \quad
         \subtypeOf{\Phi, \Delta}{\sigma}{\tau}}
        {\subtypeOf{\Phi, \Delta}{S}{\tau}} 

\infrule{\subtypeOf[k_1]{\Phi, \Delta}{\sigma_i}{\tau_i} \quad i \in 0, \ldots, n \quad\quad
         \subtypeOf{\Phi, \Delta}{\tau_r}{\sigma_r} \\
         \quad (k_0 = \mbox{-}) \lor (k_1 = \mbox{+})
        } 
        {\subtypeOf{\Phi, \Delta}
                   {\Arrow[k_0]{\tau_0, \ldots, \tau_n}{\tau_r}}
                   {\Arrow[k_1]{\sigma_0, \ldots, \sigma_n}{\sigma_r}}} 

\infrule{\subtypeOf{\Phi, \Delta}{\tau_i}{\sigma_i} & i \in 0, \ldots, n}
        {\subtypeOf{\Phi, \Delta}
          {\TApp{C}{\tau_0, \ldots, \tau_n}}
          {\TApp{C}{\sigma_0, \ldots, \sigma_n}}}

\infrule{(C : \dclass{\TApp{C}{T_0,\ldots,T_n}}{\TApp{C'}{\upsilon_0, \ldots, \upsilon_k}}{\ldots}) \in \Phi \\
         \subtypeOf{\Phi, \Delta}{\subst{\tau_0, \ldots, \tau_n}{T_0, \ldots, T_n}{\TApp{C'}{\upsilon_0, \ldots, \upsilon_k}}}{\TApp{G}{\sigma_0, \ldots, \sigma_m}}}
        {\subtypeOf{\Phi, \Delta}
          {\TApp{C}{\tau_0, \ldots, \tau_n}}
          {\TApp{G}{\sigma_0, \ldots, \sigma_m}}}



\section*{Typing}
\subsection*{Expression typing: $\yieldsOk{\Phi, \Delta, \Gamma}{e}{\opt{\tau}}{e'}{\tau'}$} 
\hrulefill\\


\sstext{ Expression typing is a relation between typing contexts, a term ($e$),
  an optional type ($\opt{\tau}$), and a type ($\tau'$).  The general idea is
  that we are typechecking a term ($e$) and want to know if it is well-typed.
  The term appears in a context, which may (or may not) impose a type constraint
  on the term.  For example, in $\dvar{x:\tau}{e}$, $e$ appears in a context
  which requires it to be a subtype of $\tau$, or to be coercable to $\tau$.
  Alternatively if $e$ appears as in $\dvar{x:\_}{e}$, then the context does not
  provide a type constraint on $e$.  This ``contextual'' type information is
  both a constraint on the term, and may also provide a source of information
  for type inference in $e$.  The optional type $\opt{\tau}$ in the typing
  relation corresponds to this contextual type information.  Viewing the
  relation algorithmically, this should be viewed as an input to the algorithm,
  along with the term.  The process of checking a term allows us to synthesize a
  precise type for the term $e$ which may be more precise than the type required
  by the context.  The type $\tau'$ in the relation represents this more
  precise, synthesized type.  This type should be thought of as an output of the
  algorithm.  It should always be the case that the synthesized (output) type is
  a subtype of the checked (input) type if the latter is present.  The
  checking/synthesis pattern allows for the propagation of type information both
  downwards and upwards. 

  It is often the case that downwards propagation is not useful.  Consequently,
  to simplify the presentation the rules which do not use the checking type
  require that it be empty ($\_$).  This does not mean that such terms cannot be
  checked when contextual type information is supplied: the first typing rule
  allows contextual type information to be dropped so that such rules apply in
  the case that we have contextual type information, subject to the contextual
  type being a supertype of the synthesized type:

}{
For subsumption, the elaboration of the underlying term carries through.
}

\infrule{\yieldsOk{\Phi, \Delta, \Gamma}{e}{\_}{e'}{\sigma} \quad\quad
         \subtypeOf{\Phi, \Delta}{\sigma}{\tau}
        }
        {\yieldsOk{\Phi, \Delta, \Gamma}{e}{\tau}{e'}{\sigma}} 

\sstext{
The implicit downcast rule also allows this when the contextual type is a
subtype of the synthesized type, corresponding to an implicit downcast.
}{
In an implicit downcast, the elaboration adds a check so that an error
will be thrown if the types do not match at runtime.
}

\infrule{\yieldsOk{\Phi, \Delta, \Gamma}{e}{\_}{e'}{\sigma} \quad\quad
         \subtypeOf{\Phi, \Delta}{\tau}{\sigma}
        }
        {\yieldsOk{\Phi, \Delta, \Gamma}{e}{\tau}{\echeck{e'}{\tau}}{\tau}} 

\sstext{Variables are typed according to their declarations:}{}

\axiom{\yieldsOk{\Phi, \Delta, \extends{\Gamma}{x}{\tau}}{x}{\_}{x}{\tau}}

\sstext{Numbers, booleans, and null all have a fixed synthesized type.}{}

\axiom{\yieldsOk{\Phi, \Delta, \Gamma}{i}{\_}{i}{\Num}}

\axiom{\yieldsOk{\Phi, \Delta, \Gamma}{\eff}{\_}{\eff}{\Bool}} 

\axiom{\yieldsOk{\Phi, \Delta, \Gamma}{\ett}{\_}{\ett}{\Bool}} 

\axiom{\yieldsOk{\Phi, \Delta, \Gamma}{\enull}{\_}{\enull}{\Bottom}} 

\sstext{A $\ethis$ expression is well-typed if we are inside of a method, and $\sigma$
is the type of the enclosing class.}{}

\infrule{\Gamma = \Gamma'_{\sigma}
        }
        {
          \yieldsOk{\Phi, \Delta, \Gamma}{\ethis}{\_}{\ethis}{\sigma}
        } 

\sstext{A fully annotated function is well-typed if its body is well-typed at its
declared return type, under the assumption that the variables have their
declared types.  
}{

A fully annotated function elaborates to a function with an elaborated body.
The rest of the function elaboration rules fill in the reified type using
contextual information if present and applicable, or $\Dynamic$ otherwise.

}

\infrule{\Gamma' = \extends{\Gamma}{\many{x}}{\many{\tau}} \quad\quad 
         \yieldsOk{\Phi, \Delta, \Gamma'}{e}{\sigma}{e'}{\sigma'}
        }
        {\yieldsOk{\Phi, \Delta, \Gamma}
                  {\elambda{\many{x:\tau}}{\sigma}{e}}
                  {\_}
                  {\elambda{\many{x:\tau}}{\sigma}{e'}}
                  {\Arrow[-]{\many{\tau}}{\sigma}}
        } 

\sstext{A function with a missing argument type is well-typed if it is well-typed with
the argument type replaced with $\Dynamic$.}
{}

\infrule{\yieldsOk{\Phi, \Delta, \Gamma}
                  {\elambda{x_0:\opt{\tau_0}, \ldots, x_i:\Dynamic, \ldots, x_n:\opt{\tau_n}}{\opt{\sigma}}{e}}
                  {\opt{\tau}}
                  {e_f}
                  {\tau_f}
        }
        {\yieldsOk{\Phi, \Delta, \Gamma}
                  {\elambda{x_0:\opt{\tau_0}, \ldots, x_i:\_, \ldots, x_n:\opt{\tau_n}}{\opt{\sigma}}{e}}
                  {\opt{\tau}}
                  {e_f}
                  {\tau_f}
        } 

\sstext{A function with a missing argument type is well-typed if it is well-typed with
the argument type replaced with the corresponding argument type from the context
type.  Note that this rule overlaps with the previous: the formal presentation
leaves this as a non-deterministic choice.}{}

\infrule{\tau_c = \Arrow[k]{\upsilon_0, \ldots, \upsilon_n}{\upsilon_r} \\
         \yieldsOk{\Phi, \Delta, \Gamma}
                  {\elambda{x_0:\opt{\tau_0}, \ldots, x_i:\upsilon_i, \ldots, x_n:\opt{\tau_n}}{\opt{\sigma}}{e}}
                  {\tau_c}
                  {e_f}
                  {\tau_f}
        }
        {\yieldsOk{\Phi, \Delta, \Gamma}
                  {\elambda{x_0:\opt{\tau_0}, \ldots, x_i:\_, \ldots, x_n:\opt{\tau_n}}{\opt{\sigma}}{e}}
                  {\tau_c}
                  {e_f}
                  {\tau_f}
        } 

\sstext{A function with a missing return type is well-typed if it is well-typed with
the return type replaced with $\Dynamic$.}{}

\infrule{\yieldsOk{\Phi, \Delta, \Gamma}
                  {\elambda{\many{x:\opt{\tau}}}{\Dynamic}{e}}
                  {\opt{\tau_c}}
                  {e_f}
                  {\tau_f}
        }
        {\yieldsOk{\Phi, \Delta, \Gamma}
                  {\elambda{\many{x:\opt{\tau}}}{\_}{e}}
                  {\opt{\tau_c}}
                  {e_f}
                  {\tau_f}
        } 

\sstext{A function with a missing return type is well-typed if it is well-typed with
the return type replaced with the corresponding return type from the context
type.  Note that this rule overlaps with the previous: the formal presentation
leaves this as a non-deterministic choice.  }{}

\infrule{\tau_c = \Arrow[k]{\upsilon_0, \ldots, \upsilon_n}{\upsilon_r} \\
         \yieldsOk{\Phi, \Delta, \Gamma}
                  {\elambda{\many{x:\opt{\tau}}}{\upsilon_r}{e}}
                  {\tau_c}
                  {e_f}
                  {\tau_f}
        }
        {\yieldsOk{\Phi, \Delta, \Gamma}
                  {\elambda{\many{x:\opt{\tau}}}{\_}{e}}
                  {\tau_c}
                  {e_f}
                  {\tau_f}
        } 


\sstext{Instance creation creates an instance of the appropriate type.}{}

% FIXME(leafp): inference
% FIXME(leafp): deal with bounds
\infrule{(C : \dclass{\TApp{C}{T_0,\ldots,T_n}}{\TApp{C'}{\upsilon_0, \ldots, \upsilon_k}}{\ldots}) \in \Phi \\ 
  \mbox{len}(\many{\tau}) = n+1}
        {\yieldsOk{\Phi, \Delta, \Gamma}
                  {\enew{C}{\many{\tau}}{}}
                  {\_}
                  {\enew{C}{\many{\tau}}{}}
                  {\TApp{C}{\many{\tau}}}
        } 


\sstext{Members of the set of primitive operations (left unspecified) can only be
applied.  Applications of primitives are well-typed if the arguments are
well-typed at the types given by the signature of the primitive.}{}

\infrule{\eprim\, :\, \Arrow[]{\many{\tau}}{\sigma} \quad\quad
         \yieldsOk{\Phi, \Delta, \Gamma}{e}{\tau}{e'}{\tau'}
        }
        {\yieldsOk{\Phi, \Delta, \Gamma}
                  {\eprimapp{\eprim}{\many{e}}}
                  {\_}
                  {\eprimapp{\eprim}{\many{e'}}}
                  {\sigma}
        } 

\sstext{Function applications are well-typed if the applicand is well-typed and has
function type, and the arguments are well-typed.}
{

Function application of an expression of function type elaborates to either a
call or a dynamic (checked) call, depending on the variance of the applicand.
If the applicand is a covariant (fuzzy) type, then a dynamic call is generated.

}

\infrule{\yieldsOk{\Phi, \Delta, \Gamma}
                  {e}
                  {\_}
                  {e'}
                  {\Arrow[k]{\many{\tau_a}}{\tau_r}} \\
         \yieldsOk{\Phi, \Delta, \Gamma}
                  {e_a}
                  {\tau_a}
                  {e_a'}
                  {\tau_a'} \quad \mbox{for}\ e_a, \tau_a \in \many{e_a}, \many{\tau_a} 
\iftrans{\\         e_c = \begin{cases}
                 \ecall{e'}{\many{e_a'}} & \text{if $k = -$}\\
                 \edcall{e'}{\many{e_a'}} & \text{if $k = +$}
                 \end{cases}}
        }
        {\yieldsOk{\Phi, \Delta, \Gamma}
                  {\ecall{e}{\many{e_a}}}
                  {\_}
                  {e_c}
                  {\tau_r}
        } 

\sstext{Application of an expression of type $\Dynamic$ is well-typed if the arguments
are well-typed at any type. }
{

  Application of an expression of type $\Dynamic$ elaborates to a dynamic call.

}

\infrule{\yieldsOk{\Phi, \Delta, \Gamma}
                  {e}
                  {\_}
                  {e'}
                  {\Dynamic} \\
         \yieldsOk{\Phi, \Delta, \Gamma}
                  {e_a}
                  {\_}
                  {e_a'}
                  {\tau_a'} \quad \mbox{for}\ e_a \in \many{e_a}
        }
        {\yieldsOk{\Phi, \Delta, \Gamma}
                  {\ecall{e}{\many{e_a}}}
                  {\_}
                  {\edcall{e'}{\many{e_a'}}}
                  {\Dynamic}
        } 

\sstext{A dynamic call expression is well-typed so long as the applicand and the
arguments are well-typed at any type.}{}

\infrule{\yieldsOk{\Phi, \Delta, \Gamma}
                  {e}
                  {\Dynamic}
                  {e'}
                  {\tau} \\
         \yieldsOk{\Phi, \Delta, \Gamma}
                  {e_a}
                  {\_}
                  {e_a'}
                  {\tau_a} \quad \mbox{for}\ e_a \in \many{e_a}
        }
        {\yieldsOk{\Phi, \Delta, \Gamma}
                  {\edcall{e}{\many{e_a}}}
                  {\_}
                  {\edcall{e'}{\many{e_a'}}}
                  {\Dynamic}
        }

\sstext{A method load is well-typed if the term is well-typed, and the method name is
present in the type of the term.}{}

\infrule{\yieldsOk{\Phi, \Delta, \Gamma}
                  {e}
                  {\_}
                  {e'}
                  {\sigma} \quad\quad
         \methodLookup{\Phi}{\sigma}{m}{\tau}
        }
        {\yieldsOk{\Phi, \Delta, \Gamma}
                  {\eload{e}{m}}
                  {\_}
                  {\eload{e'}{m}}
                  {\tau}
        }

\sstext{A method load from a term of type $\Dynamic$ is well-typed if the term is
well-typed.}
{

  A method load from a term of type $\Dynamic$ elaborates to a dynamic (checked)
  load.

}

\infrule{\yieldsOk{\Phi, \Delta, \Gamma}
                  {e}
                  {\Dynamic}
                  {e'}
                  {\tau} 
        }
        {\yieldsOk{\Phi, \Delta, \Gamma}
                  {\eload{e}{m}}
                  {\_}
                  {\edload{e'}{m}}
                  {\Dynamic}
        }

\sstext{A dynamic method load is well typed so long as the term is well-typed.}{}

\infrule{\yieldsOk{\Phi, \Delta, \Gamma}
                  {e}
                  {\Dynamic}
                  {e'}
                  {\tau} 
        }
        {\yieldsOk{\Phi, \Delta, \Gamma}
                  {\edload{e}{m}}
                  {\_}
                  {\edload{e'}{m}}
                  {\Dynamic}
        }

\sstext{A field load from $\ethis$ is well-typed if the field name is present in the
type of $\ethis$.}{}

\infrule{\Gamma = \Gamma_\tau & \fieldLookup{\Phi}{\tau}{x}{\sigma}
        }
        {\yieldsOk{\Phi, \Delta, \Gamma}
                  {\eload{\ethis}{x}}
                  {\_}
                  {\eload{\ethis}{x}}
                  {\sigma}
        } 

\sstext{An assignment expression is well-typed so long as the term is well-typed at a
type which is compatible with the type of the variable being assigned.}{}

\infrule{\yieldsOk{\Phi, \Delta, \Gamma}
                  {e}
                  {\opt{\tau}}
                  {e'}
                  {\sigma} \quad\quad
        \yieldsOk{\Phi, \Delta, \Gamma}
                  {x}
                  {\sigma}
                  {x}
                  {\sigma'}
        }
        {\yieldsOk{\Phi, \Delta, \Gamma}
                  {\eassign{x}{e}}
                  {\opt{\tau}}
                  {\eassign{x}{e'}}
                  {\sigma}
        } 

\sstext{A field assignment is well-typed if the term being assigned is well-typed, the
field name is present in the type of $\ethis$, and the declared type of the
field is compatible with the type of the expression being assigned.}{}

\infrule{\Gamma = \Gamma_\tau \quad\quad 
         \yieldsOk{\Phi, \Delta, \Gamma}
                  {e}
                  {\opt{\tau}}
                  {e'}
                  {\sigma} \\
        \fieldLookup{\Phi}{\tau}{x}{\sigma'} \quad\quad
        \subtypeOf{\Phi, \Delta}{\sigma}{\sigma'}
        }
        {\yieldsOk{\Phi, \Delta, \Gamma}
                  {\eset{\ethis}{x}{e}}
                  {\_}
                  {\eset{\ethis}{x}{e}}
                  {\sigma}
        } 

\sstext{A throw expression is well-typed at any type.}{}

\axiom{\yieldsOk{\Phi, \Delta, \Gamma}
                  {\ethrow}
                  {\_}
                  {\ethrow}
                  {\sigma}
        } 

\sstext{A cast expression is well-typed so long as the term being cast is well-typed.
The synthesized type is the cast-to type.  We require that the cast-to type be a
ground type.}{}

\comment{TODO(leafp): specify ground types}

\infrule{\yieldsOk{\Phi, \Delta, \Gamma}{e}{\_}{e'}{\sigma} \quad\quad \mbox{$\tau$ is ground}
        }
        {\yieldsOk{\Phi, \Delta, \Gamma}
                  {\eas{e}{\tau}}
                  {\_}
                  {\eas{e'}{\tau}}
                  {\tau}
        } 

\sstext{An instance check expression is well-typed if the term being checked is
well-typed. We require that the cast to-type be a ground type.}{}

\infrule{\yieldsOk{\Phi, \Delta, \Gamma}{e}{\_}{e'}{\sigma} \quad\quad \mbox{$\tau$ is ground}
        }
        {\yieldsOk{\Phi, \Delta, \Gamma}
                  {\eis{e}{\tau}}
                  {\_}
                  {\eis{e'}{\tau}}
                  {\Bool}
        } 

\sstext{A check expression is well-typed so long as the term being checked is
well-typed.  The synthesized type is the target type of the check.}{}


\infrule{\yieldsOk{\Phi, \Delta, \Gamma}{e}{\_}{e'}{\sigma}
        }
        {\yieldsOk{\Phi, \Delta, \Gamma}
                  {\echeck{e}{\tau}}
                  {\_}
                  {\echeck{e'}{\tau}}
                  {\tau}
        } 

\subsection*{Declaration typing: $\declOk[d]{\Phi, \Delta, \Gamma}{\mathit{vd}}{\mathit{vd'}}{\Gamma'}$}
\hrulefill\\

\sstext{
Variable declaration typing checks the well-formedness of the components, and
produces an output context $\Gamma'$ which contains the binding introduced by
the declaration.

A simple variable declaration with a declared type is well-typed if the
initializer for the declaration is well-typed at the declared type.  The output
context binds the variable at the declared type.
}
{
  Elaboration of declarations elaborates the underlying expressions.  
}

\infrule{\yieldsOk{\Phi, \Delta, \Gamma}{e}{\tau}{e'}{\tau'}
        }
        {\declOk[d]{\Phi, \Delta, \Gamma}
                {\dvar{x:\tau}{e}}
                {\dvar{x:\tau'}{e'}}
                {\extends{\Gamma}{x}{\tau}}
        }

\sstext{A simple variable declaration without a declared type is well-typed if the
initializer for the declaration is well-typed at any type.  The output context
binds the variable at the synthesized type (a simple form of type inference).}{}

\infrule{\yieldsOk{\Phi, \Delta, \Gamma}{e}{\_}{e'}{\tau'}
        }
        {\declOk[d]{\Phi, \Delta, \Gamma}
                {\dvar{x:\_}{e}}
                {\dvar{x:\tau'}{e'}}
                {\extends{\Gamma}{x}{\tau'}}
        }

\sstext{A function declaration is well-typed if the body of the function is well-typed
with the given return type, under the assumption that the function and its
parameters have their declared types.  The function is assumed to have a
contravariant (precise) function type.  The output context binds the function
variable only.}{}

\infrule{\tau_f = \Arrow[-]{\many{\tau_a}}{\tau_r} \quad\quad 
         \Gamma' = \extends{\Gamma}{f}{\tau_f} \quad\quad
         \Gamma'' = \extends{\Gamma'}{\many{x}}{\many{\tau_a}} \\
         \stmtOk{\Phi, \Delta, \Gamma''}{s}{\tau_r}{s'}{\Gamma_0}
        }
        {\declOk[d]{\Phi, \Delta, \Gamma}
                {\dfun{\tau_r}{f}{\many{x:\tau_a}}{s}}
                {\dfun{\tau_r}{f}{\many{x:\tau_a}}{s'}}
                {\Gamma'}
        }

\subsection*{Statement typing: $\stmtOk{\Phi, \Delta, \Gamma}{\mathit{s}}{\tau}{\mathit{s'}}{\Gamma'}$}
\hrulefill\\

\sstext{The statement typing relation checks the well-formedness of statements and
produces an output context which reflects any additional variable bindings
introduced into scope by the statements.
}{

Statement elaboration elaborates the underlying expressions.

}

\sstext{A variable declaration statement is well-typed if the variable declaration is
well-typed per the previous relation, with the corresponding output context.
}{}

\infrule{\declOk[d]{\Phi, \Delta, \Gamma}
                {\mathit{vd}}
                {\mathit{vd'}}
                {\Gamma'}
        }
        {\stmtOk{\Phi, \Delta, \Gamma}
                {\mathit{vd}}
                {\tau}
                {\mathit{vd'}}
                {\Gamma'}
        }

\sstext{An expression statement is well-typed if the expression is well-typed at any
type per the expression typing relation.}{}

\infrule{\yieldsOk{\Phi, \Delta, \Gamma}{e}{\_}{e'}{\tau}
        }
        {\stmtOk{\Phi, \Delta, \Gamma}{e}{\tau}{e'}{\Gamma}
        }

\sstext{A conditional statement is well-typed if the condition is well-typed as a
boolean, and the statements making up the two arms are well-typed.  The output
context is unchanged.}{}

\infrule{\yieldsOk{\Phi, \Delta, \Gamma}{e}{\Bool}{e'}{\sigma} \\
         \stmtOk{\Phi, \Delta, \Gamma}{s_1}{\tau_r}{s_1'}{\Gamma_1} \quad\quad
         \stmtOk{\Phi, \Delta, \Gamma}{s_2}{\tau_r}{s_2'}{\Gamma_2} 
        }
        {\stmtOk{\Phi, \Delta, \Gamma}
                {\sifthenelse{e}{s_1}{s_2}}
                {\tau_r}
                {\sifthenelse{e'}{s_1'}{s_2'}}
                {\Gamma}
        }

\sstext{A return statement is well-typed if the expression being returned is well-typed
at the given return type.  }{}

\infrule{\yieldsOk{\Phi, \Delta, \Gamma}{e}{\tau_r}{e'}{\tau}
        }
        {\stmtOk{\Phi, \Delta, \Gamma}{\sreturn{e}}{\tau_r}{\sreturn{e'}}{\Gamma}
        }

\sstext{A sequence statement is well-typed if the first component is well-typed, and the
second component is well-typed with the output context of the first component as
its input context.  The final output context is the output context of the second
component.}{}

\infrule{\stmtOk{\Phi, \Delta, \Gamma}{s_1}{\tau_r}{s_1'}{\Gamma'} \quad\quad
         \stmtOk{\Phi, \Delta, \Gamma'}{s_2}{\tau_r}{s_2'}{\Gamma''}
        }
        {\stmtOk{\Phi, \Delta, \Gamma}{s_1;s_2}{\tau_r}{s_1';s_2'}{\Gamma''}
        }


\pagebreak
\section*{Elaboration}
\setboolean{show_translation}{true}

Elaboration is a type driven translation which maps a source Dart term to a
translated term which corresponds to the original term with additional dynamic
type checks inserted to reify the static unsoundness as runtime type errors.
For the translation, we extend the source language slightly as follows.
\[
\begin{array}{lcl}
\text{Expressions $e$} & ::= & \ldots 
   \alt \edcall{e}{\many{e}} \alt \edload{e}{m} \alt \echeck{e}{\tau}\\
\end{array}
\]

The expression language is extended with an explicitly checked dynamic call
operation, and explicitly checked dynamic method load operation, and a runtime
type test.  Note that while a user level cast throws an exception on failure,
the runtime type test term introduced here produces a hard type error which
cannot be caught programmatically.

We also extend typing contexts slightly by adding an internal type to method signatures.
\[
\begin{array}{lcl}
\text{Class element ($\mathit{ce}$)} & ::= & 
  \fieldDecl{x}{\tau} \alt \methodDecl{f}{\tau}{\sigma} \\
\end{array}
\]
A method signature of the form $\methodDecl{f}{\tau}{\sigma}$ describes a method
whose public interface is described by $\sigma$, but which has an internal type
$\tau$ which is a subtype of $\sigma$, but which is properly covariant in any
type parameters.  The elaboration introduces runtime type checks to mediate
between the two types.  This is discussed further in the translation of classes
below.

\subsection*{Expression typing: $\yieldsOk{\Phi, \Delta, \Gamma}{e}{\opt{\tau}}{e'}{\tau'}$} 
\hrulefill\\


\sstext{ Expression typing is a relation between typing contexts, a term ($e$),
  an optional type ($\opt{\tau}$), and a type ($\tau'$).  The general idea is
  that we are typechecking a term ($e$) and want to know if it is well-typed.
  The term appears in a context, which may (or may not) impose a type constraint
  on the term.  For example, in $\dvar{x:\tau}{e}$, $e$ appears in a context
  which requires it to be a subtype of $\tau$, or to be coercable to $\tau$.
  Alternatively if $e$ appears as in $\dvar{x:\_}{e}$, then the context does not
  provide a type constraint on $e$.  This ``contextual'' type information is
  both a constraint on the term, and may also provide a source of information
  for type inference in $e$.  The optional type $\opt{\tau}$ in the typing
  relation corresponds to this contextual type information.  Viewing the
  relation algorithmically, this should be viewed as an input to the algorithm,
  along with the term.  The process of checking a term allows us to synthesize a
  precise type for the term $e$ which may be more precise than the type required
  by the context.  The type $\tau'$ in the relation represents this more
  precise, synthesized type.  This type should be thought of as an output of the
  algorithm.  It should always be the case that the synthesized (output) type is
  a subtype of the checked (input) type if the latter is present.  The
  checking/synthesis pattern allows for the propagation of type information both
  downwards and upwards. 

  It is often the case that downwards propagation is not useful.  Consequently,
  to simplify the presentation the rules which do not use the checking type
  require that it be empty ($\_$).  This does not mean that such terms cannot be
  checked when contextual type information is supplied: the first typing rule
  allows contextual type information to be dropped so that such rules apply in
  the case that we have contextual type information, subject to the contextual
  type being a supertype of the synthesized type:

}{
For subsumption, the elaboration of the underlying term carries through.
}

\infrule{\yieldsOk{\Phi, \Delta, \Gamma}{e}{\_}{e'}{\sigma} \quad\quad
         \subtypeOf{\Phi, \Delta}{\sigma}{\tau}
        }
        {\yieldsOk{\Phi, \Delta, \Gamma}{e}{\tau}{e'}{\sigma}} 

\sstext{
The implicit downcast rule also allows this when the contextual type is a
subtype of the synthesized type, corresponding to an implicit downcast.
}{
In an implicit downcast, the elaboration adds a check so that an error
will be thrown if the types do not match at runtime.
}

\infrule{\yieldsOk{\Phi, \Delta, \Gamma}{e}{\_}{e'}{\sigma} \quad\quad
         \subtypeOf{\Phi, \Delta}{\tau}{\sigma}
        }
        {\yieldsOk{\Phi, \Delta, \Gamma}{e}{\tau}{\echeck{e'}{\tau}}{\tau}} 

\sstext{Variables are typed according to their declarations:}{}

\axiom{\yieldsOk{\Phi, \Delta, \extends{\Gamma}{x}{\tau}}{x}{\_}{x}{\tau}}

\sstext{Numbers, booleans, and null all have a fixed synthesized type.}{}

\axiom{\yieldsOk{\Phi, \Delta, \Gamma}{i}{\_}{i}{\Num}}

\axiom{\yieldsOk{\Phi, \Delta, \Gamma}{\eff}{\_}{\eff}{\Bool}} 

\axiom{\yieldsOk{\Phi, \Delta, \Gamma}{\ett}{\_}{\ett}{\Bool}} 

\axiom{\yieldsOk{\Phi, \Delta, \Gamma}{\enull}{\_}{\enull}{\Bottom}} 

\sstext{A $\ethis$ expression is well-typed if we are inside of a method, and $\sigma$
is the type of the enclosing class.}{}

\infrule{\Gamma = \Gamma'_{\sigma}
        }
        {
          \yieldsOk{\Phi, \Delta, \Gamma}{\ethis}{\_}{\ethis}{\sigma}
        } 

\sstext{A fully annotated function is well-typed if its body is well-typed at its
declared return type, under the assumption that the variables have their
declared types.  
}{

A fully annotated function elaborates to a function with an elaborated body.
The rest of the function elaboration rules fill in the reified type using
contextual information if present and applicable, or $\Dynamic$ otherwise.

}

\infrule{\Gamma' = \extends{\Gamma}{\many{x}}{\many{\tau}} \quad\quad 
         \yieldsOk{\Phi, \Delta, \Gamma'}{e}{\sigma}{e'}{\sigma'}
        }
        {\yieldsOk{\Phi, \Delta, \Gamma}
                  {\elambda{\many{x:\tau}}{\sigma}{e}}
                  {\_}
                  {\elambda{\many{x:\tau}}{\sigma}{e'}}
                  {\Arrow[-]{\many{\tau}}{\sigma}}
        } 

\sstext{A function with a missing argument type is well-typed if it is well-typed with
the argument type replaced with $\Dynamic$.}
{}

\infrule{\yieldsOk{\Phi, \Delta, \Gamma}
                  {\elambda{x_0:\opt{\tau_0}, \ldots, x_i:\Dynamic, \ldots, x_n:\opt{\tau_n}}{\opt{\sigma}}{e}}
                  {\opt{\tau}}
                  {e_f}
                  {\tau_f}
        }
        {\yieldsOk{\Phi, \Delta, \Gamma}
                  {\elambda{x_0:\opt{\tau_0}, \ldots, x_i:\_, \ldots, x_n:\opt{\tau_n}}{\opt{\sigma}}{e}}
                  {\opt{\tau}}
                  {e_f}
                  {\tau_f}
        } 

\sstext{A function with a missing argument type is well-typed if it is well-typed with
the argument type replaced with the corresponding argument type from the context
type.  Note that this rule overlaps with the previous: the formal presentation
leaves this as a non-deterministic choice.}{}

\infrule{\tau_c = \Arrow[k]{\upsilon_0, \ldots, \upsilon_n}{\upsilon_r} \\
         \yieldsOk{\Phi, \Delta, \Gamma}
                  {\elambda{x_0:\opt{\tau_0}, \ldots, x_i:\upsilon_i, \ldots, x_n:\opt{\tau_n}}{\opt{\sigma}}{e}}
                  {\tau_c}
                  {e_f}
                  {\tau_f}
        }
        {\yieldsOk{\Phi, \Delta, \Gamma}
                  {\elambda{x_0:\opt{\tau_0}, \ldots, x_i:\_, \ldots, x_n:\opt{\tau_n}}{\opt{\sigma}}{e}}
                  {\tau_c}
                  {e_f}
                  {\tau_f}
        } 

\sstext{A function with a missing return type is well-typed if it is well-typed with
the return type replaced with $\Dynamic$.}{}

\infrule{\yieldsOk{\Phi, \Delta, \Gamma}
                  {\elambda{\many{x:\opt{\tau}}}{\Dynamic}{e}}
                  {\opt{\tau_c}}
                  {e_f}
                  {\tau_f}
        }
        {\yieldsOk{\Phi, \Delta, \Gamma}
                  {\elambda{\many{x:\opt{\tau}}}{\_}{e}}
                  {\opt{\tau_c}}
                  {e_f}
                  {\tau_f}
        } 

\sstext{A function with a missing return type is well-typed if it is well-typed with
the return type replaced with the corresponding return type from the context
type.  Note that this rule overlaps with the previous: the formal presentation
leaves this as a non-deterministic choice.  }{}

\infrule{\tau_c = \Arrow[k]{\upsilon_0, \ldots, \upsilon_n}{\upsilon_r} \\
         \yieldsOk{\Phi, \Delta, \Gamma}
                  {\elambda{\many{x:\opt{\tau}}}{\upsilon_r}{e}}
                  {\tau_c}
                  {e_f}
                  {\tau_f}
        }
        {\yieldsOk{\Phi, \Delta, \Gamma}
                  {\elambda{\many{x:\opt{\tau}}}{\_}{e}}
                  {\tau_c}
                  {e_f}
                  {\tau_f}
        } 


\sstext{Instance creation creates an instance of the appropriate type.}{}

% FIXME(leafp): inference
% FIXME(leafp): deal with bounds
\infrule{(C : \dclass{\TApp{C}{T_0,\ldots,T_n}}{\TApp{C'}{\upsilon_0, \ldots, \upsilon_k}}{\ldots}) \in \Phi \\ 
  \mbox{len}(\many{\tau}) = n+1}
        {\yieldsOk{\Phi, \Delta, \Gamma}
                  {\enew{C}{\many{\tau}}{}}
                  {\_}
                  {\enew{C}{\many{\tau}}{}}
                  {\TApp{C}{\many{\tau}}}
        } 


\sstext{Members of the set of primitive operations (left unspecified) can only be
applied.  Applications of primitives are well-typed if the arguments are
well-typed at the types given by the signature of the primitive.}{}

\infrule{\eprim\, :\, \Arrow[]{\many{\tau}}{\sigma} \quad\quad
         \yieldsOk{\Phi, \Delta, \Gamma}{e}{\tau}{e'}{\tau'}
        }
        {\yieldsOk{\Phi, \Delta, \Gamma}
                  {\eprimapp{\eprim}{\many{e}}}
                  {\_}
                  {\eprimapp{\eprim}{\many{e'}}}
                  {\sigma}
        } 

\sstext{Function applications are well-typed if the applicand is well-typed and has
function type, and the arguments are well-typed.}
{

Function application of an expression of function type elaborates to either a
call or a dynamic (checked) call, depending on the variance of the applicand.
If the applicand is a covariant (fuzzy) type, then a dynamic call is generated.

}

\infrule{\yieldsOk{\Phi, \Delta, \Gamma}
                  {e}
                  {\_}
                  {e'}
                  {\Arrow[k]{\many{\tau_a}}{\tau_r}} \\
         \yieldsOk{\Phi, \Delta, \Gamma}
                  {e_a}
                  {\tau_a}
                  {e_a'}
                  {\tau_a'} \quad \mbox{for}\ e_a, \tau_a \in \many{e_a}, \many{\tau_a} 
\iftrans{\\         e_c = \begin{cases}
                 \ecall{e'}{\many{e_a'}} & \text{if $k = -$}\\
                 \edcall{e'}{\many{e_a'}} & \text{if $k = +$}
                 \end{cases}}
        }
        {\yieldsOk{\Phi, \Delta, \Gamma}
                  {\ecall{e}{\many{e_a}}}
                  {\_}
                  {e_c}
                  {\tau_r}
        } 

\sstext{Application of an expression of type $\Dynamic$ is well-typed if the arguments
are well-typed at any type. }
{

  Application of an expression of type $\Dynamic$ elaborates to a dynamic call.

}

\infrule{\yieldsOk{\Phi, \Delta, \Gamma}
                  {e}
                  {\_}
                  {e'}
                  {\Dynamic} \\
         \yieldsOk{\Phi, \Delta, \Gamma}
                  {e_a}
                  {\_}
                  {e_a'}
                  {\tau_a'} \quad \mbox{for}\ e_a \in \many{e_a}
        }
        {\yieldsOk{\Phi, \Delta, \Gamma}
                  {\ecall{e}{\many{e_a}}}
                  {\_}
                  {\edcall{e'}{\many{e_a'}}}
                  {\Dynamic}
        } 

\sstext{A dynamic call expression is well-typed so long as the applicand and the
arguments are well-typed at any type.}{}

\infrule{\yieldsOk{\Phi, \Delta, \Gamma}
                  {e}
                  {\Dynamic}
                  {e'}
                  {\tau} \\
         \yieldsOk{\Phi, \Delta, \Gamma}
                  {e_a}
                  {\_}
                  {e_a'}
                  {\tau_a} \quad \mbox{for}\ e_a \in \many{e_a}
        }
        {\yieldsOk{\Phi, \Delta, \Gamma}
                  {\edcall{e}{\many{e_a}}}
                  {\_}
                  {\edcall{e'}{\many{e_a'}}}
                  {\Dynamic}
        }

\sstext{A method load is well-typed if the term is well-typed, and the method name is
present in the type of the term.}{}

\infrule{\yieldsOk{\Phi, \Delta, \Gamma}
                  {e}
                  {\_}
                  {e'}
                  {\sigma} \quad\quad
         \methodLookup{\Phi}{\sigma}{m}{\tau}
        }
        {\yieldsOk{\Phi, \Delta, \Gamma}
                  {\eload{e}{m}}
                  {\_}
                  {\eload{e'}{m}}
                  {\tau}
        }

\sstext{A method load from a term of type $\Dynamic$ is well-typed if the term is
well-typed.}
{

  A method load from a term of type $\Dynamic$ elaborates to a dynamic (checked)
  load.

}

\infrule{\yieldsOk{\Phi, \Delta, \Gamma}
                  {e}
                  {\Dynamic}
                  {e'}
                  {\tau} 
        }
        {\yieldsOk{\Phi, \Delta, \Gamma}
                  {\eload{e}{m}}
                  {\_}
                  {\edload{e'}{m}}
                  {\Dynamic}
        }

\sstext{A dynamic method load is well typed so long as the term is well-typed.}{}

\infrule{\yieldsOk{\Phi, \Delta, \Gamma}
                  {e}
                  {\Dynamic}
                  {e'}
                  {\tau} 
        }
        {\yieldsOk{\Phi, \Delta, \Gamma}
                  {\edload{e}{m}}
                  {\_}
                  {\edload{e'}{m}}
                  {\Dynamic}
        }

\sstext{A field load from $\ethis$ is well-typed if the field name is present in the
type of $\ethis$.}{}

\infrule{\Gamma = \Gamma_\tau & \fieldLookup{\Phi}{\tau}{x}{\sigma}
        }
        {\yieldsOk{\Phi, \Delta, \Gamma}
                  {\eload{\ethis}{x}}
                  {\_}
                  {\eload{\ethis}{x}}
                  {\sigma}
        } 

\sstext{An assignment expression is well-typed so long as the term is well-typed at a
type which is compatible with the type of the variable being assigned.}{}

\infrule{\yieldsOk{\Phi, \Delta, \Gamma}
                  {e}
                  {\opt{\tau}}
                  {e'}
                  {\sigma} \quad\quad
        \yieldsOk{\Phi, \Delta, \Gamma}
                  {x}
                  {\sigma}
                  {x}
                  {\sigma'}
        }
        {\yieldsOk{\Phi, \Delta, \Gamma}
                  {\eassign{x}{e}}
                  {\opt{\tau}}
                  {\eassign{x}{e'}}
                  {\sigma}
        } 

\sstext{A field assignment is well-typed if the term being assigned is well-typed, the
field name is present in the type of $\ethis$, and the declared type of the
field is compatible with the type of the expression being assigned.}{}

\infrule{\Gamma = \Gamma_\tau \quad\quad 
         \yieldsOk{\Phi, \Delta, \Gamma}
                  {e}
                  {\opt{\tau}}
                  {e'}
                  {\sigma} \\
        \fieldLookup{\Phi}{\tau}{x}{\sigma'} \quad\quad
        \subtypeOf{\Phi, \Delta}{\sigma}{\sigma'}
        }
        {\yieldsOk{\Phi, \Delta, \Gamma}
                  {\eset{\ethis}{x}{e}}
                  {\_}
                  {\eset{\ethis}{x}{e}}
                  {\sigma}
        } 

\sstext{A throw expression is well-typed at any type.}{}

\axiom{\yieldsOk{\Phi, \Delta, \Gamma}
                  {\ethrow}
                  {\_}
                  {\ethrow}
                  {\sigma}
        } 

\sstext{A cast expression is well-typed so long as the term being cast is well-typed.
The synthesized type is the cast-to type.  We require that the cast-to type be a
ground type.}{}

\comment{TODO(leafp): specify ground types}

\infrule{\yieldsOk{\Phi, \Delta, \Gamma}{e}{\_}{e'}{\sigma} \quad\quad \mbox{$\tau$ is ground}
        }
        {\yieldsOk{\Phi, \Delta, \Gamma}
                  {\eas{e}{\tau}}
                  {\_}
                  {\eas{e'}{\tau}}
                  {\tau}
        } 

\sstext{An instance check expression is well-typed if the term being checked is
well-typed. We require that the cast to-type be a ground type.}{}

\infrule{\yieldsOk{\Phi, \Delta, \Gamma}{e}{\_}{e'}{\sigma} \quad\quad \mbox{$\tau$ is ground}
        }
        {\yieldsOk{\Phi, \Delta, \Gamma}
                  {\eis{e}{\tau}}
                  {\_}
                  {\eis{e'}{\tau}}
                  {\Bool}
        } 

\sstext{A check expression is well-typed so long as the term being checked is
well-typed.  The synthesized type is the target type of the check.}{}


\infrule{\yieldsOk{\Phi, \Delta, \Gamma}{e}{\_}{e'}{\sigma}
        }
        {\yieldsOk{\Phi, \Delta, \Gamma}
                  {\echeck{e}{\tau}}
                  {\_}
                  {\echeck{e'}{\tau}}
                  {\tau}
        } 

\subsection*{Declaration typing: $\declOk[d]{\Phi, \Delta, \Gamma}{\mathit{vd}}{\mathit{vd'}}{\Gamma'}$}
\hrulefill\\

\sstext{
Variable declaration typing checks the well-formedness of the components, and
produces an output context $\Gamma'$ which contains the binding introduced by
the declaration.

A simple variable declaration with a declared type is well-typed if the
initializer for the declaration is well-typed at the declared type.  The output
context binds the variable at the declared type.
}
{
  Elaboration of declarations elaborates the underlying expressions.  
}

\infrule{\yieldsOk{\Phi, \Delta, \Gamma}{e}{\tau}{e'}{\tau'}
        }
        {\declOk[d]{\Phi, \Delta, \Gamma}
                {\dvar{x:\tau}{e}}
                {\dvar{x:\tau'}{e'}}
                {\extends{\Gamma}{x}{\tau}}
        }

\sstext{A simple variable declaration without a declared type is well-typed if the
initializer for the declaration is well-typed at any type.  The output context
binds the variable at the synthesized type (a simple form of type inference).}{}

\infrule{\yieldsOk{\Phi, \Delta, \Gamma}{e}{\_}{e'}{\tau'}
        }
        {\declOk[d]{\Phi, \Delta, \Gamma}
                {\dvar{x:\_}{e}}
                {\dvar{x:\tau'}{e'}}
                {\extends{\Gamma}{x}{\tau'}}
        }

\sstext{A function declaration is well-typed if the body of the function is well-typed
with the given return type, under the assumption that the function and its
parameters have their declared types.  The function is assumed to have a
contravariant (precise) function type.  The output context binds the function
variable only.}{}

\infrule{\tau_f = \Arrow[-]{\many{\tau_a}}{\tau_r} \quad\quad 
         \Gamma' = \extends{\Gamma}{f}{\tau_f} \quad\quad
         \Gamma'' = \extends{\Gamma'}{\many{x}}{\many{\tau_a}} \\
         \stmtOk{\Phi, \Delta, \Gamma''}{s}{\tau_r}{s'}{\Gamma_0}
        }
        {\declOk[d]{\Phi, \Delta, \Gamma}
                {\dfun{\tau_r}{f}{\many{x:\tau_a}}{s}}
                {\dfun{\tau_r}{f}{\many{x:\tau_a}}{s'}}
                {\Gamma'}
        }

\subsection*{Statement typing: $\stmtOk{\Phi, \Delta, \Gamma}{\mathit{s}}{\tau}{\mathit{s'}}{\Gamma'}$}
\hrulefill\\

\sstext{The statement typing relation checks the well-formedness of statements and
produces an output context which reflects any additional variable bindings
introduced into scope by the statements.
}{

Statement elaboration elaborates the underlying expressions.

}

\sstext{A variable declaration statement is well-typed if the variable declaration is
well-typed per the previous relation, with the corresponding output context.
}{}

\infrule{\declOk[d]{\Phi, \Delta, \Gamma}
                {\mathit{vd}}
                {\mathit{vd'}}
                {\Gamma'}
        }
        {\stmtOk{\Phi, \Delta, \Gamma}
                {\mathit{vd}}
                {\tau}
                {\mathit{vd'}}
                {\Gamma'}
        }

\sstext{An expression statement is well-typed if the expression is well-typed at any
type per the expression typing relation.}{}

\infrule{\yieldsOk{\Phi, \Delta, \Gamma}{e}{\_}{e'}{\tau}
        }
        {\stmtOk{\Phi, \Delta, \Gamma}{e}{\tau}{e'}{\Gamma}
        }

\sstext{A conditional statement is well-typed if the condition is well-typed as a
boolean, and the statements making up the two arms are well-typed.  The output
context is unchanged.}{}

\infrule{\yieldsOk{\Phi, \Delta, \Gamma}{e}{\Bool}{e'}{\sigma} \\
         \stmtOk{\Phi, \Delta, \Gamma}{s_1}{\tau_r}{s_1'}{\Gamma_1} \quad\quad
         \stmtOk{\Phi, \Delta, \Gamma}{s_2}{\tau_r}{s_2'}{\Gamma_2} 
        }
        {\stmtOk{\Phi, \Delta, \Gamma}
                {\sifthenelse{e}{s_1}{s_2}}
                {\tau_r}
                {\sifthenelse{e'}{s_1'}{s_2'}}
                {\Gamma}
        }

\sstext{A return statement is well-typed if the expression being returned is well-typed
at the given return type.  }{}

\infrule{\yieldsOk{\Phi, \Delta, \Gamma}{e}{\tau_r}{e'}{\tau}
        }
        {\stmtOk{\Phi, \Delta, \Gamma}{\sreturn{e}}{\tau_r}{\sreturn{e'}}{\Gamma}
        }

\sstext{A sequence statement is well-typed if the first component is well-typed, and the
second component is well-typed with the output context of the first component as
its input context.  The final output context is the output context of the second
component.}{}

\infrule{\stmtOk{\Phi, \Delta, \Gamma}{s_1}{\tau_r}{s_1'}{\Gamma'} \quad\quad
         \stmtOk{\Phi, \Delta, \Gamma'}{s_2}{\tau_r}{s_2'}{\Gamma''}
        }
        {\stmtOk{\Phi, \Delta, \Gamma}{s_1;s_2}{\tau_r}{s_1';s_2'}{\Gamma''}
        }



\end{document}
